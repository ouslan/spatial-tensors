\documentclass[12pt]{article}
\usepackage[margin=1in]{geometry}
\usepackage{setspace}
\usepackage{graphicx}
\usepackage{float}
\usepackage{amsmath}
\usepackage{booktabs}
\usepackage{hyperref}

\title{Modeling Spatial Dependence: A Simulation-Based Comparison of Parametric and Semi-Parametric Approaches}
\author{
  Alejandro M. Ouslan \\
  \small{University of Puerto Rico, Mayaguez} \\
  \small{\texttt{alejandro.ouslan@upr.edu}}
}
\date{May 14, 2025}

\begin{document}

\maketitle


\begin{abstract}
	\textit{
		This research aims to compare the performance of spatial regression models that rely on predefined
		weight matrices with that of semi-parametric regression models using spatial smoothers. The preliminary
		results show that the best preforming model is the Queens
		model as this is the model used to generate the data. The second best preforming
		model is the penalized tensor products. This shows that if there is no reasonable
		argument for picking the spatial weights matrix the penalized tensor products are a
		reasonable starting point for controlling for spatial variability.
	}
\end{abstract}

\noindent{\underline{\textbf{Keywords:}}}\; Spatial simulation, Spatial Regressions, GAMs, Tensor Products \\


\noindent{\underline{\textbf{Acknowledgements:}}}\; This research was sponsored by the Mathematics department
of the University of Puerto Rico, Mayaguez.


\end{document}
